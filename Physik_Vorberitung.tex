\documentclass{scrartcl}
\usepackage{amsmath,amssymb,amstext}
\usepackage{mathtools}
\usepackage[ngerman]{babel}
\usepackage[final]{pdfpages}
\selectlanguage{german}

\begin{document}
    \section*{Selbstinduktion}
        Selbstinduktion hast du immer, wenn du irgendwo eine Spannung induzierst. Die Teilchen erzeugen dadurch, dass Sie sich bewegen eine Spannung gegen die vorher induzierte Spannung.
    \section*{Hall-Effekt}
        magnetische Flussdichte kann mit einem Voltmeter gemessen werden
    \section*{Formel die wichtig sind für LK}
        \subsection*{Lorentzkraft}
            \begin{equation}
                \begin{multlined}
                    F\textsubscript{L}=B*Q*v
                \end{multlined}
            \end{equation}
        \subsection*{Radialkraft}
            \begin{equation}
                \begin{multlined}
                    F\textsubscript{R}=\frac{mv^2}{r}
                \end{multlined}
            \end{equation}
        \subsection*{Zyklotron}
            \begin{equation}
                \begin{multlined}
                    F\textsubscript{L}=F\textsubscript{R} \\
                    \text{Nach Geschwindigkeit umstellen, Ansatz von Lorentzkraft} \\
                    B*Q = \frac{mv}{r} \| * \frac{r}{m} \\
                    v=\frac{B*Q*r}{m} \\
                    v=\frac{2\pi*r}{T} \\
                    \frac{2\pi*r}{T}=\frac{B*Q*r}{m} \| Reziproke bilden \\
                    \frac{T}{2\pi*r}=\frac{m}{B*Q*r} \| *2\pi*r \\
                    T=\frac{2\pi*m}{B*Q} \\
                    f=\frac{B*Q}{2\pi*m} \\
                    \omega=2\pi*f \\
                \end{multlined}
            \end{equation}
        \subsection*{Aufgabe zum Zyklotron}
        Geg:
        \begin{equation}
            \begin{multlined}
                U\textsubscript{B} = 1250V \\
                Lithium-7-ion=4m\textsubscript{n}+3m\textsubscript{p} \\
                r\textsubscript{$\max$} = 1,3m
                f=8*10^5 Hz
            \end{multlined}
        \end{equation}
        Ges:
        \begin{equation}
            \begin{multlined}
                T,
                v\textsubscript{End},
                v\textsubscript{1},
                E\textsubscript{ges},
                E\textsubscript{1},
                n
            \end{multlined}
        \end{equation}
        Lsg:
        \begin{equation}
            \begin{multlined}
                T = \frac{1}{f} = \frac{1}{8*10^5 Hz} = 1,25 * 10\textsuperscript{-6}s \\
                v\textsubscript{End} = \frac{2\pi*r\textsubscript{$\max$}}{T} = \frac{2\pi*1,3m}{1,25 * 10\textsuperscript{-6}s} = 6.534.512 \frac{m}{s} \\
                v=\sqrt{\frac{2*Q*U}{m}} \\
                v\textsubscript{1}=\sqrt{\frac{2e*1250V}{4m\textsubscript{n}+3m\textsubscript{p}}} = 184.886 \frac{m}{s} \\
                E\textsubscript{1}= Q*U = 1250 eV = 2*10\textsuperscript{-6} J \\
            \end{multlined}
        \end{equation}
\end{document}